\documentclass{article}
\usepackage{hyperref}
\begin{document}
\title{Detailed documentation of the library for special mathematical functions by W. Fullerton}
\author{Arjen Markus}
\maketitle

\section{Introduction}

The original code to the library can be found on \url{http://netlib.org}. It has been updated to the current
Fortran standard:

\begin{itemize}
\item
The code has been transformed to free-form.
\item
Elimination of \verb+GOTO+ and introduction of \verb+IMPLICIT NONE+. In some cases calculations for
particular cases has been moved to make the control flow easier.
\item
Error messages because of arguments out of range have been replaced by \verb+NaN+ of \verb+Inf+. The original code would
generally stop the program.
\item
Warnings because of loss of accuracy have been removed, as such messages, written to the screen, do not seem
appropriate for a general library. It may be useful to add an auxiliary routine that checks the arguments.
\item
All code is now contained in a small set of modules, where the module \verb+specfunc_fullerton+ is the overall
module that gives access to all public functions.
\end{itemize}

For the moment only the single-precision implementation has been updated.

\section{Airy functions}
The module \verb+fullerton_airy+ contains the routines for evaluating the Airy Ai and Bi functions and their
first derivatives.

The Airy functions are solutions to the differential equation:

\begin{equation}
    \frac{d^2y}{dx^2} - xy = 0
\end{equation}

The implementations of Airy Ai and Bi functions are:

\begin{verbatim}
    y = airy_ai(x)
    y = airy_bi(x)
\end{verbatim}

The implementations of the first derivatives are:

\begin{verbatim}
    y = airy_aiprime(x)
    y = airy_biprime(x)
\end{verbatim}

\emph{Notes:}
\begin{itemize}
\item
The function $Ai$ decays rapidly. For large values of $x$ the implementation returns 0.
\item
The function $Bi$ grows rapidly. For large values of $x$ the implementation returns \verb~+Inf~. Similarly for its derivative.
\end{itemize}


\section{Modified Bessel functions}
The module \verb+fullerton_bessel+ implements the modified Bessel functions of the first and second kind, $I_0$, $I_1$, $K_0$, $K_1$
and $I_n$, $K_n$. The ordinary Bessel functions are part of the current Fortran standard.

\begin{verbatim}
    y = bessel_i0(x)
    y = bessel_i1(x)
    y = bessel_k0(x)
    y = bessel_k1(x)
    y = bessel_in(n,x)
    y = bessel_kn(n,x)
\end{verbatim}
The argument $x$ must be positive and $n$ a non-negative integer.

\emph{Notes:}
\begin{itemize}
\item
The general modified Bessel functions with the integer parameter $n$ have been implemented via
well-known recursive relations.
\item
The functions $I_0$, $I_1$, $I_n$ may overflow if the value of $x$ is too large, \verb~+Inf~ is returned in that case.
\item
The functions $K_0$, $K_1$, $K_n$ may underflow if the value of $x$ is too large, \verb~0~ is returned in that case.
\end{itemize}


\section{Beta function and related functions}
The module \verb+fullerton_beta+ contains implementations for the beta function, the incomplete beta function
and the logarithm of the beta function.

The beta function is defined as:

\begin{equation}
    B(p,q) = \int_0^1 x^{p-1} (1 - x)^{q-1} dx = \frac{\Gamma(p)\Gamma(q)}{\Gamma(p+q)}
\end{equation}
\noindent where parameters $p, q > -1$

The incomplete beta function is defined as:

\begin{equation}
    B(x;p,q) = \int_0^x u^{p-1} (1 - u)^{q-1} du
\end{equation}

The implementations of these functions and the logarithm are:

\begin{verbatim}
    y = beta(p,q)
    y = beta_inc(x,p,q)
    y = log_beta(p,q)
\end{verbatim}

\emph{Notes:}
\begin{itemize}
\item
The implementation of the Beta function requires $a$ and $b$ to be both positive. This is a limitation with
respect to the actual mathematical function.
\item
For very large values of $a$ and $b$ the implementation simply returns 0.
\end{itemize}


\section{Exponential integrals}
The module \verb+fullerton_ei+ can be used to evaluate a variety of indefinite integrals:

\begin{eqnarray}
    E_n(x) &=& \int^\infty_1 \frac{e^{-xt}}{t^n} dt              \\
    Ei(x)  &=& -E_1(-x)                                          \\
    Ci(x)  &=& -\int^\infty_x \frac{\cos t}{t} dt                \\
    Si(x)  &=& -\int^x_0 \frac{\sin t}{t} dt                     \\
    Chi(x) &=& \gamma + \ln x + \int^x_0 \frac{\cosh t -1}{t} dt \\
    Shi(x) &=& \int^x_0 \frac{\sinh t -1}{t} dt
\end{eqnarray}
\noindent where $\gamma$ is the Euler-Mascheroni constant, 0.5772156649...

The exponential integrals $Ei$, $E_1$ and $E_n$ are implemented as:

\begin{verbatim}
    y = integral_ei(x)
    y = integral_e1(x)
    y = integral_en(n,x)
\end{verbatim}

The cosine integral $Ci$, sine integral $Si$, hyperbolic cosine integral $Chi$ and
hyperbolic sine integral $Shi$ are implemented as:

\begin{verbatim}
    y = integral_ci(x)
    y = integral_chi(x)
    y = integral_si(x)
    y = integral_shi(x)
\end{verbatim}

\emph{Notes:}
\begin{itemize}
\item
For the logarithmic integral $Li$ the argument $x$ must be positive, otherwise \verb+NaN+ is returned.
\item
Also for $x \approx 1$ the result of $Li$ has low precision.
\item
THe argument $x$ must not be zero for the exponential integral $E_1$.
\item
The implementation of $E_1$ returns 0 for very large arguments.
\end{itemize}


\section{Gamma function and related functions}
The module \verb+fullerton_gamma+ implements the Gamma function and several related functions. Note that
the Gamma function in this module is an alternative to the standard function and is used by among others the
module \verb+fullerton_beta+ to ensure consistent results.

The implemented functions are:
\begin{eqnarray}
    \Gamma(x)   &=& \int^\infty_0 t^{x-1} e^{-t} dt \\
    \gamma(a,x) &=& \int^x_0 t^{a-1} e^{-t} dt      \\
    \Gamma(a,x) &=& \int^\infty_x t^{a-1} e^{-t} dt \\
    \Psi(x)     &=& \frac{d}{dx} \ln \Gamma(x)      \\
    \gamma^*(x) &=& \frac{x^{-a}}{\Gamma(x)} \int^x_0 t^{a-1} e^{-t} dt
\end{eqnarray}
\noindent where $\Psi$ is the digamma function and $\gamma^*$ is the incomplete Tricomi gamma function (see, for instance,
\url{https://www.cs.purdue.edu/homes/wxg/selected_works/section_02/155.pdf}.

In addition the module evaluates the logarithm of the Gamma function and the reciprocal. A direct evaluation
of these expressions is more accurate than evaluating the Gamma function and then taking the logarithm or the reciprocal
of that result.

The functions are implemented as:
\begin{verbatim}
    y = fgamma(x)             ! alternative to the intrinsic function,
                              ! used internally
    y = log_gamma(x)
    y = inc_gamma(a,x)        ! gamma(a,x)
    y = inc_gamma_compl(a,x)  ! Gamma(a,x), the complement
    y = gamma_tricomi(x)
    y = reciprocal_gamma(x)
    y = digamma(x)
\end{verbatim}

The subroutine \verb+sign_log_gamma+ evaluates the logarithm and the sign of the Gamma function:
\begin{verbatim}
    call sign_log_gamma( x, algam, sgngam )
\end{verbatim}

\emph{Notes:}
\begin{itemize}
\item
For the Gamma function and its logarithm overflow may occur if the argument is too large or is around zero or a negative integer.
In that case \verb~+Inf~ is returned (regardless of the sign of the mathematical Gamma function in that neighbourhood).
\item
Also for $x$ near a negative integer the result has low precision.
\item
For the incomplete Gamma functions, the value of $x$ must be non-negative. For $x = 0$ and $a \leq 0$, these functions are
not defined. In these cases \verb+NaN+ is returned.
\item
The Tricomi gamma function requires $x$ to be non-negative.
\item
For the digamma function $psi$ the result is inaccurate if $x$ is close to zero or a negative integer (poles of $\Gamma(x)$).
\end{itemize}


\section{Inverse exponential integrals}
The module \verb+fullerton_inv_ci+ implements two inverse functions: the inverse cosine integral
and the inverse hyperbolic cosine integral:
\begin{eqnarray}
    y &=& Ci^{-1}(x) \\
    y &=& Chi^{-1}(x)
\end{eqnarray}

These mathematical functions are implemented as:
\begin{verbatim}
    y = inverse_ci(x)
    y = inverse_chi(x)
\end{verbatim}

\emph{Notes:}
\begin{itemize}
\item
The implementation of $Chi^{-1}(x)$ presents a discontinuity around $x = 3$. This is
quite unusual within this library. It needs to be further investigated.
\item
For very small values of $x$ the implementations of $Ci$ and $Chi$ return 0, instead of an underflowing number.
\end{itemize}

\section{Pochhammer symbol and related funcions}
The module \verb+fullerton_poch+ implements the Pochhammer symbol and two related functions:
\begin{eqnarray}
    (a)_x             &=& \frac{\Gamma(a+x)}{\Gamma(a)} \\
    \frac{(a)_x-1}{x} &=& \frac{\Gamma(a+x)-1}{ x \Gamma(a)} \\
    n!                &=& n \cdot (n-1) \cdot (n-2) \cdot ... \cdot 2 \cdot 1
\end{eqnarray}
The second function is meant for special situations where the argument x is much smaller than 1.
The third function is the ordinary factorial, but as the value can exceed the range of ordinary integers,
it is calculated as a real number, extending the range of the argument.

The functions are implemented as:
\begin{verbatim}
    y = poch(a,x)
    y = poch1(a,x)
    y = fac(n)
\end{verbatim}

\emph{Notes:}
\begin{itemize}
\item
For the factorial the argument must be zero or positive, otherwise \verb+Nan+ is returned.
\item
If the argument for the factorial is too large, \verb~+Inf~ is returned.
\item
For the Pochhammer symbol, the exceptions are fairly complicated:
    \begin{itemize}
    \item If the sum $a+x$ is a negative integer, but $a$ is not, the value is undefined.
    \item For large values of $x$ the result is not accurate, because the sum $a+x$ can not be calculated accurately.
    \item This is also the case if $a$ or $a+x$ are close to a negative integer.
    \end{itemize}

\end{itemize}


\section{Miscelleaneous functions}
The module \verb+fullerton_misc+ defines as number of miscellaneous functions:

\begin{itemize}
\item
The cubic root of a real number (including negative numbers)
\item
The functions $f(x) = ln(1+x)$ and $g(x) = (e^x-1)/x$ with the argument $|x| << 1$
\item
Dawson's integral:
\begin{equation}
     D(x) = e^{-x^2} \int^y_0 e^{y^2} dy
\end{equation}

\noindent and its first derivative $D'(x) = 1 - 2 x D(x)$.
\item
Spence's function:
\begin{equation}
     S(x) = - \int^x_0 \ln |1-y| dy
\end{equation}
\noindent though the values as evaluated and the description on \url{https://mathworld.wolfram.com} are
not in agreement. This is something to be examined.
\end{itemize}

These functions are implemented as:
\begin{verbatim}
    y = cbrt(x)
    y = lnrel(x)
    y = exprel(x)
    y = integral_dawson(x)
    y = dawson_prime(x)
    y = integral_spence(x)
\end{verbatim}

\emph{Notes:}
\begin{itemize}
\item
For the implementation \verb+lnrel+ the argument must be larger than -1.
\item
The implementation \verb+lnrel+ will have low precision if the argument is too near -1.
\item
Also for $x \approx 1$ the result has low precision.
\end{itemize}


\section{Complex versions}
The module \verb+fullerton_complex+ offers the implementation of complex versions of several of the above functions:
\begin{itemize}
\item The beta function and the logarithm of the beta function
\item The gamma function, reciprocal gamma function and the logarithm of the gamma function
\item The digamma function
\item Several miscellaneous functions, $\arctan2(z,w)$, $\log_{10}(z)$, $(e^{z}-1)/z$, $\ln(1+z)$, $z^{1/3}$
\end{itemize}

The names of the implementations are the same as the real versions, notably the complex gamma function is implemented.
as \verb+fgamma(z0)+.

\end{document}
